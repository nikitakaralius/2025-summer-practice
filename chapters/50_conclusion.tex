\chapter*{Заключение} \label{ch:general-conclusion}
\addcontentsline{toc}{chapter}{Заключение}

В ходе производственной практики были разработаны и апробированы подходы к стандартизации CI/CD процессов и миграции репозиториев с GitHub на GitLab.

Была проведена комплексная аналитическая работа по изучению существующей инфраструктуры компании,
что позволило выделить 26 кластеров репозиториев,
требующих различных подходов к миграции.
Классификация позволила определить объем и сложность работ, выявить типовые и уникальные проекты.

Также был разработан подход стандартизации с использованием компонентов GitLab,
который обеспечивает переиспользование логики конвейеров.
Были выделены и реализованы два ключевых типа компонентов: компонентные конвейеры и компонентные шаблоны,
что позволило создать гибкую и масштабируемую систему.

Параллельно был разработан автоматизированный инструмент для миграции,
основанный на принципах конечного автомата.
Трехфазный подход обеспечил контролируемый и безопасный процесс миграции с минимальным влиянием на рабочие процессы разработчиков.

Далее разработанные решения были применены при переносе Ansible Role репозиториев.
Среднее время миграции одного типового репозитория составило 15--30 минут.

Во время эксплуатации также стали явными некоторые проблемы, которые можно улучшить.

Например, для решения проблемы с устаревшими ссылками в еще не перенесенных репозиториях
можно разработать автоматизированный инструмент их автоматического поиска и обновления.

Кроме того, в компании пока что не появились инструкции и хорошие практики по написанию конвейеров.
Этот недочет планируется улучшить, как только добавление и редактирование конвейеров станет доступно другим командам.
