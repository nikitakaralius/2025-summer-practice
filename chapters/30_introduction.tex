\chapter*{Введение}
\addcontentsline{toc}{chapter}{Введение}

Системы контроля версий - это одни из самых важных инструментов в современной разработке.
На текущий момент самой популярной системой является git.
Как следует из названия этот инструмент позволяет управлять версиями исходного кода продукта.
Для разработчика-одиночки такого функционала может быть достаточно, однако для большой команды нужен общей сервер,
единая точка синхронизации, где разработчики смогут совместно работать над проектом.
Для этого на основе git было разработано множество платформ, таких как GitHub, GitLab и BitBucket.
Помимо хранения кода и управления его версиями, эти площадки предоставляют возможности управления задачами,
а также позволяют организовать процессы непрерывной интеграции и доставки (CI/CD).

Компания ООО Mindbox до недавнего времени использовала GitHub для разработки своего продукта.
Однако из-за изменения политических условий возникли риски использования этой площадки.
GitHub не предоставляет self-hosted решение для компаний - вся информация хранится на серверах этой компании.
Соответственно GitHub, следуя определенным правилам, может безвозвратно заблокировать организацию Mindbox,
тем самым лишив ее доступа к исходному коду, а также надолго прервать все процессы компании.

В связи с этим было принято решение превентивно перевести все репозитории и процессы организации на self-hosted GitLab сервер.
Основная сложность такой задачи лежит не только в количестве репозиториев для переноса,
но и наличии разных от проекта к проекту CI/CD шаблонов, описанных на основе GitHub Actions.

Целей переезда было поставлено несколько:
\begin{enumerate}
  \item Перенести все 800 репозиториев компании вместе с CI/CD шаблонами без остановки процесса разработки;
  \item Стандартизировать CI/CD шаблоны для упрощения массового внедрения нововведений (например, анализа уязвимостей).
\end{enumerate}

Для реализации этих целей были выделены следующие задачи:
\begin{enumerate}
  \item Изучить GitLab CI/CD, основные отличия от GitHub Actions и опыт миграции с GitHub на GitLab других компаний;
  \item Собрать информацию о репозиториях компании и кластеризовать их для дальнейшей стандартизации CI/CD шаблонов;
  \item Стандартизировать пайплайны типовых репозиториев;
  \item Спроектировать и реализовать приложение миграции репозиториев;
  \item Перенести часть репозиториев с помощью разработанного инструмента.
\end{enumerate}
