\chapter{Некоторые детали реализации миграции} \label{ch:ch3}
В этой главе будут описаны некоторые моменты реализации, которые потребовали отдельного внимания и нестандартных подходов.

\section{Замена динамических матриц} \label{sec:dymamic-matricies}
В разделе \enquote{Выявление ключевых различий между GitHub Actions и GitLab CI/CD}\cite{sec:gh-and-gl-differences} была
отмечена проблема отсутствия динамических матриц, которые были нужны для множества компонентов.
Решением проблемы стало использование нисходящих конвейеров (downstream pipelines).
Этот механизм позволяет запускать дочерний конвейер из основного, передавая ему текст конфигурации.
Таким образом, удалось воспроизвести динамическую матрицу в GitLab по следующему принципу:
родительский конвейер получает динамические данные, например, путь к папке по её названию,
запрашивает статическую конфигурацию дочернего конвейера у GitLab,
после чего с помощью утилиты \texttt{envsubst} заменяет статические переменные на динамические данные,
и запускает дочерний конвейер с нужными параметрами.

\section{Оптимизация типовых задач} \label{sec:common-tasks-optimization}
Помимо отсутствия части функционала как в GitLab, команда столкнулась с ухудшением производительности в перенесенных конвейерах.
Одним из способов оптимизации стал специальный Docker образ, содержащий набор часто используемых утилит.
До этого в каждом конвейере утилита устанавливалась отдельно через curl или wget при каждом запуске.
Заранее подготовленный образ оказался эффективнее за счет возможности кэширование слоев в исполнителях GitLab.

\section{Выводы} \label{sec:conclusion}
