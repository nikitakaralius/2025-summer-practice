\chapter{Реализация компонентов и приложения-мигратора} \label{ch:ch3}
После того как команда определилась с процессом миграции, параллельно началась разработка как компонентов, так и приложения-мигратора.
В этой главе будут приведены листинги с их реализацией и пояснениями.

\section{Реализация компонентов} \label{sec:components-impl}
Изначально большая часть ресурса была брошена на разработку стандартизированных пайплайнов — тех самых компонентов в терминологии GitLab.
Такое распределение отчасти связано с тем, что миграцией занималась команда C\# разработчиков, которые не столь активно до этого работали
с bash скриптами\cite{bash} и инфраструктурными инструментами.

Тем не менее во время написания компонентов выделились два вида компонентов, речь о которых пойдет далее.
Стоит отметить, что выработавшиеся подходы и названия являются терминологией внутри компании,
потому что GitLab в документации не давал обозначения тем вещам, которые мы упоминали часто, но для них не было слова.

\subsection{Компонентный конвейер} \label{subsec:component-template}


\section{Выводы} \label{sec:conclusion}
