\chapter{Название третьей главы: разработка программного обеспечения} \label{ch3}

% к реализации (из главы компоненты)
Для всех задач есть набор часто используемых утилит, таких как \texttt{jq}, \texttt{curl}, \texttt{git}, которые необходимы почти в каждом конвейере.
Они были добавлены в отдельный Docker образ, который используется в задачах по мере необходимости.
Поскольку слои образов кешируются в исполнителях CI/CD, скачивание образов происходит быстрее, чем их ручная установка в каждой задаче.

Хорошим стилем является наличие введения к главе. Во введении может быть описана цель написания главы, а также приведена краткая структура главы.

\section{Название параграфа} \label{ch3:sec1}

\section{Название параграфа} \label{ch3:sec2}

%\FloatBarrier % заставить рисунки и другие подвижные (float) элементы остановиться


\section{Выводы} \label{ch3:conclusion}

Текст выводов по главе \thechapter.


%% Вспомогательные команды - Additional commands
%
%\newpage % принудительное начало с новой страницы, использовать только в конце раздела
%\clearpage % осуществляется пакетом <<placeins>> в пределах секций
%\newpage\leavevmode\thispagestyle{empty}\newpage % 100 % начало новой страницы
