\chapter{Апробация стандартизированных компонентов и автоматизированного процесса переноса} \label{ch:ch4}
После того как были написаны все шаги для миграции, началось активное использование мигратора для переноса репозиториев.
По окончании первой фазы в проектах подключались только что созданные стандартизированные конвейеры.

\section{Перенос Ansible Role репозиториев} \label{sec:ansible-role-migration}
Одним из этапов миграции стал переезд репозиториев, содержащих Ansible роли\cite{ansible-roles}.
В части написанного функционала мигратор прекрасно отработал и позволил перенести 44 репозитория за 2 недели руками 1 человека.
В среднем, перенос одного такого репозитория занимал от 15 до 30 минут.
Однако также возникли некоторые проблемы:
\begin{itemize}
  \item Команда, отвечавшая за Ansible репозитории, попросила отключить уведомления о переносе, потому что они им мешали.
        Для этого пришлось доработать изначальный функционал.
        Позже с подобной просьбой стали обращаться и другие команды.
        По сути функционал оповещений оказался не только бесполезным, но и раздражительным.
  \item В еще неперенесенных репозиториях могли оставаться ссылки на перенесенный.
        В таком случае необходимо было при помощи поиска находить все такие места и исправлять их вручную.
  \item На части проектов конвейеры могли изначально не работать.
        Команда решила не ждать исправлений и не вносить их самостоятельно для подобных случаев.
        К счастью, архитектура мигратора была выстроена так, что позволяла пропускать определенные шаги, хотя
        такое поведение изначально и не предполагалось.
\end{itemize}

\section{Перенос уникальных репозиториев} \label{sec:custom-repo-translation}
Помимо типовых репозиториев, для которых можно было легко подключить стандартизированные конвейеры,
в компании был ряд проектов со сложными уникальными шаблонами.
На подобных проектах мигратор тоже себя хорошо показал, однако сам процесс мог занимать до нескольких недель
из-за ручного написания конвейеров.

\section{Выводы} \label{sec:conclusion}
В рамках апробации было проверено приложение-мигратор, а также стандартизированные конвейеры.
Их совокупность позволяла достаточно быстро, хоть и не без проблем, переносить типовые репозитории.
Однако же для уникальных проектов мигратор не прибавлял скорость к переносу в процентном соотношении.
