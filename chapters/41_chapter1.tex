\chapter{Изучение и анализ возможности переезда и опыта других компаний} \label{ch:ch1}
Непосредственно перед самим переездом стоит понять насколько эта задача выполнима.
Кроме того, несмотря на любовь программистов изобретать велосипеды,
первоначально стоит ознакомиться с готовыми решениями, а также с опытом других компаний,
которые уже осуществляли подобный перенос.
С них и начнем.


\section{Опыт других компаний}\label{sec:other-companies-expirience}
Самокат и Aviasales - две сопоставимые с Mindbox крупные российские компании,
о процессах переезда на GitLab которых хорошо известно из статей и конференций.

\subsection{Самокат}\label{subsec:samokat-experience}
%%

\subsection{Aviasales}\label{subsec:aviasales-experience}
%%


\section{Существующие решения}\label{sec:existing-solutions}
GitLab имеет встроенный инструмент переезда репозиториев из других сервисов,
который полностью переносит коммиты, пользователей, ветки, заметки, а также частично настройки репозиториев.
Хоть эта разработка и облегчает перенос, она все же имеет ряд критических ограничений:
\begin{itemize}
  \item Не умеет транслировать CI/CD шаблоны GitHub Actions в GitLab Pipelines;
  \item Не переносит многие настройки из GitHub: политики доступа, правила слияния и т.д.
\end{itemize}
Как отправная точка инструмент неплохо себя зарекомендовал, его можно взять за основу решения для Mindbox.
