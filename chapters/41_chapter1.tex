\chapter{Изучение и анализ возможности переезда и опыта других компаний} \label{ch:ch1}
Непосредственно перед самим переездом стоит понять насколько эта задача выполнима.
Кроме того, несмотря на любовь программистов изобретать велосипеды,
первоначально стоит ознакомиться с готовыми решениями, а также с опытом других компаний,
которые уже осуществляли подобный перенос.
С них и стоит начать.


\section{Опыт других компаний}\label{sec:other-companies-expirience}
Изучив доступные материалы по переезду компаний на GitLab,
наблюдается интересная картина: подробным опытом миграции целой компании поделился преимущественно ecom.tech (Samokat),
в то время как большинство других ресурсов ограничиваются описанием базовых принципов переноса репозиториев.

Однако начальные условия у Самоката были другие.
Они использовали три разрозненных GitLab, в которых хоть и по-разному, но уже были описаны шаблоны на языке платформы.
Из интересного — компания показала свой план миграции, в котором также фигурирует стандартизация CI/CD шаблонов.
Как они это будут делать — не рассказали, однако дали свою оценку для переноса 500 репозиториев, что сопоставимо с количеством в Mindbox.
Основная же часть доклада посвящена способам воссоздания функциональных возможностей платной версии GitLab, потому что
Самокат не может приобрести лицензию.
Представленные Самокатом способы ухудшают опыт использования разработчиками по сравнению с GitHub,
что является не допустимо для Mindbox.
Кроме того, компания может позволить себе приобрести платную лицензию.

Отдельно стоит отметить статью "Импортозамещение облаков: как настроить GitLab Runner в Yandex Cloud и не обанкротиться".
В отличие от GitHub GitLab подразумевает настройку машин, на которых будет запускаться пайплайн.
Mindbox в своей инфраструктуре активно использует Yandex Cloud, поэтому, и хоть в данном отчете и не будет описана организация машин-исполнителей,
текст статьи оказался очень полезным.

К сожалению, в ходе исследования не удалось найти опыт других больших компаний, связанных с переносом репозиториев
и CI/CD шаблонов на GitLab.
Подобная ситуация может быть обусловлена следующими факторами:
\begin{itemize}
  \item Многие компании в РФ изначально выбирали self-hosted GitLab, поэтому задачи переезда не появлялось.
  \item Новизна и трудность задачи: причина переезда появилась недавно, компании еще не успели завершить перенос, соответственно рассказывать об опыте пока рано.
\end{itemize}

\section{Существующие решения}\label{sec:existing-solutions}
Далее будут разобраны существующие решения по переносу, которые были описаны во множестве статей.
GitLab имеет встроенный инструмент переезда репозиториев из других сервисов,
который полностью переносит коммиты, пользователей, ветки, заметки, а также частично настройки репозиториев.
Хоть эта разработка и облегчает перенос, она все же имеет ряд критических ограничений:
\begin{itemize}
  \item Не умеет транслировать CI/CD шаблоны GitHub Actions в GitLab Pipelines;
  \item Не переносит многие настройки из GitHub: политики доступа, правила слияния и т.д.
\end{itemize}
Как отправная точка инструмент неплохо себя зарекомендовал, его можно взять за основу решения для Mindbox.
