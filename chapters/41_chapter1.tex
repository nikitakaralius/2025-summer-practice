\chapter{Название первой главы: всестороннее изучение объекта и предмета исследования, анализ результатов, полученных другими авторами} \label{ch1}

% не рекомендуется использовать отдельную section <<введение>> после лета 2020 года
%\section{Введение. Сложносоставное название первого параграфа первой главы для~демонстрации переноса слов в содержании} \label{ch1:intro}

Хорошим стилем является наличие введения к главе, которое \textit{начинается непосредственно после названия главы, без оформления в виде отдельного параграфа}. Во введении может быть описана цель написания главы, а также приведена краткая структура главы. Например, в параграфе \ref{ch1:sec1} приведены примеры оформления одиночных формул, рисунков и таблицы. Параграф \ref{ch1:sec2} посвящён многострочным формулам и сложносоставным рисункам.

Текст данной главы призван привести \textit{краткие} примеры оформления текстово-графических объектов. Более подробные примеры можно посмотреть в следующей главе, а также в рекомендациях студентам \cite{spbpu-student-thesis-template-author-guide}.


\section{Название параграфа} \label{ch1:sec1}


\subsection{Название первого подпараграфа первого параграфа первой главы для~демонстрации переноса слов в содержании} % ~ нужен, чтобы избавиться от висячего предлога (союза) в конце строки

Содержание первого подпараграфа первого параграфа первой главы.



Одиночные формулы оформляют в окружении \texttt{equation}, например, как указано в следующей одиночной нумерованной формуле:
%
%
\begin{equation}% лучше не оставлять пропущенную строку (\par) перед окружениями для избежания лишних отсупов в pdf
\label{eq:Pi-ch1} % eq - equations, далее название, ch поставлено для избежания дублирования
\pi \approx 3,141.
\end{equation}
%
%
\begin{figure}[ht!]
	\center
	\caption{Вид на гидробашню СПбПУ \cite{spbpu-gallery}}
	\label{fig:spbpu_hydrotower}
\end{figure}
%
%
%\begin{table} [htbp]% Пример оформления таблицы
%	\centering\small
%	\caption{Представление данных для сквозного примера по ВКР \cite{Peskov2004}}%
%	\label{tab:ToyCompare}
%		\begin{tabular}{|l|l|l|l|l|l|}
%			\hline
%			$G$&$m_1$&$m_2$&$m_3$&$m_4$&$K$\\
%			\hline
%			$g_1$&0&1&1&0&1\\ \hline
%			$g_2$&1&2&0&1&1\\ \hline
%			$g_3$&0&1&0&1&1\\ \hline
%			$g_4$&1&2&1&0&2\\ \hline
%			$g_5$&1&1&0&1&2\\ \hline
%			$g_6$&1&1&1&2&2\\ \hline
%		\end{tabular}
%	\normalsize% возвращаем шрифт к нормальному
%\end{table}


% \firef{} от figure reference
% \taref{} от table reference
% \eqref{} от equation reference

На \firef{fig:spbpu_hydrotower} изображена гидробашня СПбПУ, а в \taref{tab:ToyCompare} приведены данные, на примере которых коротко и наглядно будет изложена суть ВКР.


\section{Название параграфа} \label{ch1:sec2}



Формулы могут быть размещены в несколько строк. Чтобы выставить номер формулы напротив средней строки, используйте окружение \verb|multlined| из пакета \verb|mathtools| следующим образом \cite{Ganter1999}:
%
\begin{equation}
\label{eq:fConcept-order-ch1}
\begin{multlined}
(A_1,B_1)\leq (A_2,B_2)\; \Leftrightarrow \\  \Leftrightarrow\; A_1\subseteq A_2\; \Leftrightarrow \\ \Leftrightarrow\; B_2\subseteq B_1.
\end{multlined}
\end{equation}


Используя команду \verb|\labelcref| из пакета \verb|cleveref|, допустимо следующим образом оформлять ссылку на несколько формул:
(\labelcref{eq:Pi-ch1,eq:fConcept-order-ch1}).
%
%

Пример ссылок \cite{Article,Book,Booklet,Conference,Inbook,Incollection,Manual,Mastersthesis,Misc,Phdthesis,Proceedings,Techreport,Unpublished,badiou:briefings}, а также ссылок с указанием страниц, на котором отображены номера страниц  \cite[с.~96]{Naidenova2017} или в виде мультицитаты на несколько источников \cites[с.~96]{Naidenova2017}[с.~46]{Ganter1999}. Часть библиографических записей носит иллюстративный характер и не имеет отношения к реальной литературе.



%\FloatBarrier % заставить рисунки и другие подвижные (float) элементы остановиться

\section{Выводы} \label{ch1:conclusion}

Текст выводов по главе \thechapter.

Кроме названия параграфа <<выводы>> можно использовать (единообразно по всем главам) следующие подходы к именованию последних разделов с результатами по главам:
\begin{itemize}
	\item <<выводы по главе N>>, где N --- номер соответствующей главы;
	\item <<резюме>>;
	\item <<резюме по главе N>>, где N --- номер соответствующей главы.
\end{itemize}

Параграф с изложением выводов по главе \textit{является обязательным}.

%% Вспомогательные команды - Additional commands
%
%\newpage % принудительное начало с новой страницы, использовать только в конце раздела
%\clearpage % осуществляется пакетом <<placeins>> в пределах секций
%\newpage\leavevmode\thispagestyle{empty}\newpage % 100 % начало новой страницы
