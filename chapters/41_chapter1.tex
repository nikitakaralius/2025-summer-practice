\chapter{Изучение и анализ возможности переезда и опыта других компаний} \label{ch:ch1}
Непосредственно перед самим переездом стоит понять насколько эта задача выполнима.
Кроме того, полезно будет ознакомиться с готовыми решениями, а также с опытом других компаний, которые уже осуществляли подобный перенос.

\section{Опыт других компаний}\label{sec:other-companies-expirience}
Изучив доступные материалы по переезду компаний на GitLab,
наблюдается интересная картина: подробным опытом миграции целой компании поделился преимущественно \enquote{ecom.tech}
(также известна как \enquote{Самокат})\cite{samokat-conference-talk},
в то время как большинство других ресурсов ограничиваются описанием базовых принципов переноса репозиториев.

Однако начальные условия у Самоката были другие.
Перед переездом они уже использовали три разрозненных GitLab, в которых хоть и по-разному, но уже были описаны шаблоны на языке платформы.
Из интересного — компания показала свой план миграции, в котором также фигурирует стандартизация CI/CD шаблонов.
Как они это будут делать — не рассказали, однако дали свою оценку трудозатрат переноса 500 репозиториев, что сопоставимо с количеством в Mindbox.
Основная же часть доклада посвящена способам воссоздания функциональных возможностей платной версии GitLab, потому что у
Самоката нет возможности приобрести лицензию из-за санкционных ограничений (компания зарегистрирована в РФ).
К сожалению, эта часть доклада не оказалась полезной,
ввиду того, что представленные Самокатом способы ухудшают опыт использования инструмента разработчиками по сравнению с GitHub,
что является не допустимо для Mindbox.
Кроме того, компания может позволить себе приобрести платную лицензию.

Отдельно стоит отметить статью \enquote{Импортозамещение облаков: как настроить GitLab Runner в Yandex Cloud и не обанкротиться}\cite{yc-runners-article}.
В отличие от GitHub GitLab подразумевает настройку машин, на которых будет запускаться пайплайн~\cite{gl-runners}.
Mindbox как раз в своей инфраструктуре активно использует Yandex Cloud\cite{yc}, и хоть в данном отчете и не будет описана организация машин-исполнителей,
текст статьи оказался очень полезным.

\section{Существующие решения}\label{sec:existing-solutions}
\subsection{Официальные инструменты} \label{subsec:official-migration-instruments}
После исследования опыта людей, следует разобрать существующие решения.

GitLab имеет встроенный инструмент переноса репозиториев из других сервисов\cite{official-gl-migrator}, который полностью переносит пользователей сервиса, коммиты, ветки.
На первый взгляд кажется, что эта разработка сможет обеспечить перенос, однако она имеет ряд критических ограничений:
\begin{itemize}
  \item Не умеет транслировать CI/CD шаблоны GitHub Actions в GitLab Pipelines;
  \item Не переносит многие настройки из GitHub: правила слияния, политики редактирования и т.д.
\end{itemize}
Тем не менее как отправная точка инструмент неплохо себя зарекомендовал, было принято взять его за основу решения для Mindbox.

\subsection{Сторонние решения} \label{subsec:third-party-migration-instruments}
Так как GitLab не предоставляет встроенные инструменты трансляции CI/CD шаблонов, стоит поискать сторонние.
Инструмент \texttt{gh-actions-importer}\cite{gh-actions-importer} представляет собой единственное публично доступное и активно поддерживаемое решение для работы с GitHub и GitLab.
Он обладает возможностью частичного преобразования конфигураций конвейеров, однако работает исключительно в одном направлении — с GitLab CI/CD на GitHub Actions,
что не удовлетворяет надобностям компании.

\section{Возможные причины отсутствия готовых решений} \label{sec:instrument-absence}

К сожалению, в ходе исследования не удалось найти опыт других крупных компаний и инструментов, связанных с переносом репозиториев и CI/CD шаблонов из GitHub в GitLab.
Подобная ситуация может быть обусловлена следующими факторами:
\begin{enumerate}
  \item Многие компании, которым важно было иметь self-hosted решение и полный контроль над своим кодом, используют GitLab с самого начала.
        Соответственно, нужды переезда у них не возникало.

  \item Массовая миграция с GitHub на GitLab — это одноразовая задача.
        Сторонним бизнесам невыгодно делать решения для таких задач, потому что они плохо масштабируются.

  \item Настройки CI/CD в различных организациях значительно отличаются и могут применять различные стратегии,
        механизмы и параметры, а также внешние интеграции.
        Грубо говоря, невозможно сделать что-то универсальное для всех сразу.
\end{enumerate}
