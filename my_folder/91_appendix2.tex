\chapter{Некоторые дополнительные примеры}\label{appendix-extra-examples}							% 

В приложении приведены формулы \eqref{eq:Pi-app}, \eqref{eq:Pi-app-}, \firef{fig:spbpu_hydrotower-app}, \firef{fig:spbpu_hydrotower-app-}, \taref{tab:ToyCompare-app}, \taref{tab:ToyCompare-app-}\footnote{Внимание! Пример оформления подстрочной ссылки (сноски).}.


\begin{equation}% лучше не оставлять пропущенную строку (\par) перед окружениями для избежания лишних отсупов в pdf
\label{eq:Pi-app-} % eq - equations, далее название, ch поставлено для избежания дублирования
\pi \approx 3,141.
\end{equation}
%
%
\begin{figure}[ht!] 
	\center
	\includegraphics [scale=0.27] {my_folder/images//spbpu_hydrotower}
	\caption{Вид на гидробашню СПбПУ \cite{spbpu-gallery}} 
	\label{fig:spbpu_hydrotower-app-}  
\end{figure}

\begin{table} [htbp]% Пример оформления таблицы
	\centering\small
	\caption{Представление данных для сквозного примера по ВКР \cite{Peskov2004}}%
	\label{tab:ToyCompare-app-}		
	\begin{tabular}{|l|l|l|l|l|l|}
		\hline
		$G$&$m_1$&$m_2$&$m_3$&$m_4$&$K$\\
		\hline
		$g_1$&0&1&1&0&1\\ \hline
		$g_2$&1&2&0&1&1\\ \hline
		$g_3$&0&1&0&1&1\\ \hline
		$g_4$&1&2&1&0&2\\ \hline
		$g_5$&1&1&0&1&2\\ \hline
		$g_6$&1&1&1&2&2\\ \hline		
	\end{tabular}	
	\normalsize% возвращаем шрифт к нормальному
\end{table}




\section{Подраздел приложения}\label{app-2-1}							


\begin{equation}% лучше не оставлять пропущенную строку (\par) перед окружениями для избежания лишних отсупов в pdf
\label{eq:Pi-app} % eq - equations, далее название, ch поставлено для избежания дублирования
\pi \approx 3,141.
\end{equation}
%
%
\begin{figure}[ht!] 
	\center
	\includegraphics [scale=0.27] {my_folder/images//spbpu_hydrotower}
	\caption{Вид на гидробашню СПбПУ \cite{spbpu-gallery}} 
	\label{fig:spbpu_hydrotower-app}  
\end{figure}

\begin{table} [htbp]% Пример оформления таблицы
	\centering\small
	\caption{Представление данных для сквозного примера по ВКР \cite{Peskov2004}}%
	\label{tab:ToyCompare-app}		
	\begin{tabular}{|l|l|l|l|l|l|}
		\hline
		$G$&$m_1$&$m_2$&$m_3$&$m_4$&$K$\\
		\hline
		$g_1$&0&1&1&0&1\\ \hline
		$g_2$&1&2&0&1&1\\ \hline
		$g_3$&0&1&0&1&1\\ \hline
		$g_4$&1&2&1&0&2\\ \hline
		$g_5$&1&1&0&1&2\\ \hline
		$g_6$&1&1&1&2&2\\ \hline		
	\end{tabular}	
	\normalsize% возвращаем шрифт к нормальному
\end{table}

