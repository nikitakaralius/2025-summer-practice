\chapter{Название второй главы: разработка метода, алгоритма, модели исследования} \label{ch2}
	
% не рекомендуется использовать отдельную section <<введение>> после лета 2020 года
%\section{Введение} \label{ch2:intro}

Глава посвящена более подробным примерам оформления текстово-графических объектов.

В параграфе \ref{ch2:title-abbr} приведены примеры оформления многострочной формулы и одиночного рисунка. Параграф \ref{ch2:sec-abbr} раскрывает правила оформления перечислений и псевдокода. В параграфе \ref{ch2:sec-very-short-title} приведены примеры оформления сложносоставных рисунков, длинных таблиц, а также теоремоподобных окружений.


\section{Название параграфа} \label{ch2:title-abbr} %название по-русски



%%%%
%%		
%%  \input{...} commands are used only to sychronize some parts of the text with the author guide. Authors are free to type the text directly in .tex-files   
%%  \input{...} комманды используются только, чтобы синхронизировать части текта с рекомендациями авторам. Авторы  вольны вносить текст непосредственно в файл главы  
%%  
 %% ВНИМАНИЕ: для того, чтобы избежать лишнего отступа между текстом  и формулами, пожалуйста, начинайте формулы без пропуска строки в исходном коде как в строках #2 и #3.
	Все формулы, размещенные в отдельных строках, подлежат нумерации, например, как формулы \eqref{eq:UpArrow-G} и \eqref{eq:DownArrow-G} из \cite{Ganter1999}.
	\begin{align}
	\label{eq:UpArrow-G}
	& A\uA =  \{ m\in{}M\:|\:gIm\:\forall  g \in{} A \}; \\
	\label{eq:DownArrow-G}
	& B\dA =  \{ g\in{}G\:|\:gIm\:\forall  m \in{} B \}.
	\end{align}

Обратим внимание, что формулы содержат знаки препинания и что они выровнены по левому краю (с помощью знака \verb|&| окружения \texttt{align}).
 % пример двух выравнивания двух формул в окружении align


На \firef{fig:spbpu-new-bld-autumn-ch2} приведёна фотография Нового научно-исследовательского корпуса СПбПУ.

	\begin{figure}[ht] 
	\center
	\includegraphics [scale=0.27] {my_folder/images/spbpu_new_bld_autumn}
	\caption{Новый научно-исследовательский корпус СПбПУ \cite{spbpu-gallery}} 
	\label{fig:spbpu-new-bld-autumn-ch2}  
	\end{figure}
	


	
\section{Название параграфа} \label{ch2:sec-abbr} %название по-русски
	
Название параграфа оформляется с помощью команды \verb|\section{...}|, название главы --- \verb|\chapter{...}|. 
	

\subsection{Название подпараграфа} \label{ch2:subsec-title-abbr} %название по-русски


Название подпараграфа оформляется с помощью команды  \texttt{\textbackslash{}subsection\{...\}}.


%\subsubsection{Название подподпараграфа} \label{ch2:subsubsec-title-abbr} %название по-русски
	
Использование подподпараграфов в основной части крайне не рекомендуется. В случае использования, необходимо вынести данный номер в содержание.	
Название подпараграфа оформляется с помощью команды  \texttt{\textbackslash{}subsubsecti\-on\{...\}}.



Вместо подподпараграфов рекомендовано использовать перечисления, если это не мешает логике работы.

Перечисления могут быть с нумерационной частью и без неё и использоваться с иерархией и без иерархии. Нумерационная часть при этом формируется следующим способом:

\begin{enumerate}[1.]
	\item в перечислениях {\itshape без иерархии} оформляется арабскими цифрами с точкой (или длинным тире).
	\item В перечислениях {\itshape с иерархией} --- в последовательности сначала прописных латинских букв с точкой, затем арабских цифр с точкой и далее --- строчных латинских букв со скобкой.
\end{enumerate}


%% Если в дальнейшем нужно сделать сслыку на один из элементов нумеруемого перечисления, то нужно использовать конктрукцию типа:

%\begin{enumerate}[label=\arabic{enumi}.,ref=\arabic{enumi}]
%	\item text 1 \label{item:text1}
%	\item text 2
%\end{enumerate}
%\ref{item:text1}.


Далее приведён пример перечислений с иерархией.


\begin{enumerate}
	\item Первый пункт.
	\item Второй пункт.
	\item Третий пункт.
	\item По ГОСТ 2.105--95 \cite{gost-russian-text-documents} первый уровень нумерации идёт буквами русского или латинского алфавитов ({\itshape для определенности выбираем английский алфавит}),
	а второй "--- цифрами.
	\begin{enumerate}
		\item В данном пункте лежит следующий нумерованный список:
		\begin{enumerate}
			\item первый пункт;
			\item третий уровень нумерации не нормирован ГОСТ 2.105--95 ({\itshape для определенности выбираем английский алфавит});
			\item обращаем внимание на строчность букв в этом нумерованном и следующем маркированном списке:
			\begin{itemize}
				\item первый пункт маркированного списка.
			\end{itemize}
		\end{enumerate}
	\end{enumerate}
	\item Пятый пункт верхнего уровня перечисления.
\end{enumerate}

Маркированный список (без нумерационной части) используется, если нет необходимости ссылки на определенное положение в списке:
\begin{itemize}
	\item первый пункт c {\itshape маленькой буквы} по правилам русского языка;
	\item второй пункт c {\itshape маленькой буквы} по правилам русского языка.
\end{itemize}
 % правила использования перечислений	

	
Оформление псевдокода необходимо осуществлять с помощью пакета \verb|algorithm2e| в окружении \verb|algorithm|. Данное окружение интерпретируется в шаблоне как рисунок. Пример оформления псевдокода алгоритма приведён на \firef{alg:AlgoFDSCALING}. 
	
	
	\begin{algorithm} %[h]
		\SetKwFunction{algoDTestsFDSCALING}{}
		\SetKwProg{myalg}{Algorithm}{}{} %write in 2nd agrument <<Algorithm>>, <<Procedure>> etc
		\nonl\myalg{\algoDTestsFDSCALING}{
			\KwInput{the many-valued context $\cont[M]\eqdef(G,M,W,J)$, the class membership $\epsilon: G\to K$}
			\KwOutput{positive and negative binary contexts $\overbar{\cont[K]_+}\eqdef(\overbar{G_+},M,I_+)$, $\overbar{\cont[K]_-}\eqdef(\overbar{G_-},M,I_-)$ such that i-tests found in $\overbar{\cont[K]_+}$ are diagnostic tests in $\cont[M]$, and objects from $\overbar{\cont[K]_-}$ are counter-examples} %последние строки формируют начальное множество диагностических тестов
			\For {$\forall g_i,$ $g_j \in G$\label{step:FD-scaling-first-step}}{
				%(\tcp*[f]{possible inlined comment})
				\If{$i < j$ }{
					$\overbar{G} \leftarrow (g_i,g_j)$\;
				}
			}
			%		$M\leftarrow M\setminus k$\;
			\For {$\forall (g_i,g_j)\in \overbar{G}$}{
				%(\tcp*[f]{possible inlined comment})
				\If{$m(g_i) = m(g_j)$ }{ %на самом деле здесь цикл по всем компонентам вектора-строки
					$(g_i,g_j) I m$\; % or setI() function
				}
				\uIf{$\epsilon(g_i) = \epsilon(g_j)$ }{
					$\overbar{G_+} \leftarrow (g_i,g_j)$\;
				}
				\lElse{$\overbar{G_-} \leftarrow (g_i,g_j)$\label{FD-scaling-step-last}}
			}
			$I_+= I\cap (\overbar{G_+}\times M)$, $I_-= I\cap (\overbar{G_-}\times M)$\label{FD-scaling-step-newK}\;
			\For {$\forall \overbar{g_+}\in \overbar{G_+}$, $\forall \overbar{g_-}\in \overbar{G_-}$ }{
				\If{$\overbar{g_+}\uA \subseteq \overbar{g_-}\uA$ }{
					$\overbar{G_+} \leftarrow \overbar{G_+} \setminus \overbar{g_+}$\;
				}
			}
			%		\Return \;
		}
		\caption{Псевдокод алгоритма \texttt{DiagnosticTestsScalingAndInferring} \cite{Naidenova2017}}\label{alg:AlgoFDSCALING}
		% example of adding an item to Index
		% \index for accepted papers only
		\index[ru]{алгоритм!\texttt{название\_алгоритма}}
		% key words <<алгоритм>> и <<algorithm>> keep unmodified
		\index[en]{algorithm!\texttt{algorighm\_title}}
		% authors can used the key word <<процедура>> (procedure) и т.п.
		%
		%
	    % another example:
		\index[ru]{алгоритм!\texttt{DiagnosticTestsScaling\-AndInferring}} %нужен ручной перенос \- из-за ошибки в MakeIndex для команды \texttt
		%ключевые слова <<алгоритм>> и <<algorithm>> не менять
		\index[en]{algorithm!\texttt{DiagnosticTestsScaling\-AndInferring}} %нужен ручной перенос \- из-за ошибки в MakeIndex для команды \texttt
	\end{algorithm}

	% another example of adding an arbitrary keyword to Index
	% some useful keywords: theorem, proposition, lemma, equation etc
	% please, use short keywords (2-3 max)
	\index[ru]{длинное-название-возможное-например-на-немецком} % длинные названия первого уровня как правило запрещены
	\index[en]{long-title-possible-for-example-in-German}

Обратим внимание, что можно сослаться на строчку \ref{step:FD-scaling-first-step} псевдокода из \firef{alg:AlgoFDSCALING}.
 % пример оформления псевдокода алгоритма 	

	
\section{Название параграфа} \label{ch2:sec-very-short-title} %название по-русски


	
%% ВНИМАНИЕ: для того, чтобы избежать лишнего отступа между текстом  и формулами, пожалуйста, начинайте формулы без пропуска строки в исходном коде как в строках #2 и #3.
Одиночные формулы также, как и отдельные формулы в составе группы, могут быть размещены в несколько строк. Чтобы выставить номер формулы напротив средней строки, используйте окружение \verb|multlined| из пакета \verb|mathtools| следующим образом \cite{Ganter1999}:
\begin{equation} % \tag{S} % tag - вписывает свой текст
\label{eq:fConcept-order-G}
\begin{multlined}
(A_1,B_1)\leq (A_2,B_2)\; \Leftrightarrow \\  \Leftrightarrow\; A_1\subseteq A_2\; \Leftrightarrow \\ \Leftrightarrow\; B_2\subseteq B_1.
\end{multlined}
\end{equation}


Используя команду \verb|\labelcref{...}| из пакета \verb|cleveref|, допустимо оформить ссылку на несколько формул, например, (\labelcref{eq:UpArrow-G,eq:DownArrow-G,eq:fConcept-order-G}).
 % пример оформления одиночной формулы в несколько строк

Пример оформления четырёх иллюстраций в одном текстово-графическом объекте приведён на \firef{fig:spbpu_sc-four-photos}. Это возможно благодаря использованию пакета \verb|subcaption|.

\begin{figure}[ht]
	\adjustbox{minipage=1.3em,valign=t}{\subcaption{}\label{fig:spbpu_sc-a}}%
	\begin{subfigure}[t]{\dimexpr.5\linewidth-1.3em\relax}
		\centering
		\includegraphics[width=.95\linewidth,valign=t]{my_folder/images/spbpu_sc_system}
	\end{subfigure}
\hfill %выровнять по ширине
	\adjustbox{minipage=1.3em,valign=t}{\subcaption{}\label{fig:spbpu_sc-b}}%
	\begin{subfigure}[t]{\dimexpr.5\linewidth-1.3em\relax}
		\centering
		\includegraphics[width=.95\linewidth,valign=t]{my_folder/images/spbpu_sc_refr}
	\end{subfigure}
\\[20pt]
	\adjustbox{minipage=1.3em,valign=t}{\subcaption{}\label{fig:spbpu_sc-c}}%
\begin{subfigure}[t]{\dimexpr.5\linewidth-1.3em\relax}
	\centering
	\includegraphics[width=.95\linewidth,valign=t]{my_folder/images/spbpu_sc_hall}
\end{subfigure}%
\hfill %выровнять по ширине
\adjustbox{minipage=1.3em,valign=t}{\subcaption{}\label{fig:spbpu_sc-d}}%
\begin{subfigure}[t]{\dimexpr.5\linewidth-1.3em\relax}
	\centering
	\includegraphics[width=.95\linewidth,valign=t]{my_folder/images/spbpu_sc_box}
\end{subfigure}
\captionsetup{justification=centering} %центрировать
\caption{Фотографии суперкомпьютерного центра СПбПУ \cite{spbpu-gallery}: {\itshape a} --- система хранения данных и узлы NUMA-вычислителя; {\itshape b} --- холодильные машины на крыше научно-исследовательского корпуса; {\itshape c} --- машинный зал; {\itshape d} --- элементы вычислительных устройств}
\label{fig:spbpu_sc-four-photos}
\end{figure}

Далее можно ссылаться на составные части данного рисунка как на самостоятельные объекты: \firef{fig:spbpu_sc-a}, \firef{fig:spbpu_sc-b}, \firef{fig:spbpu_sc-c}, \firef{fig:spbpu_sc-d} или на три из четырёх изображений одновременно: рис.\labelcref{fig:spbpu_sc-a,fig:spbpu_sc-b,fig:spbpu_sc-c}.
 % пример подключения 4х иллюстраций в одном рисунке

%На \firef{fig:spbpu_whitehall-three-photos} приведены три картинки под~общим номером и~названием, но с раздельной нумерацией подрисунков посредством пакета \verb|subcaption|.
%
\begin{figure}[!htbp]
	\adjustbox{minipage=1.3em,valign=t}{\subcaption{}\label{fig:spbpu_whitehall-a}}%
	\begin{subfigure}[t]{\dimexpr.3\linewidth-1.3em\relax}
		\centering
		\includegraphics[width=.95\linewidth,valign=t]{my_folder/images//spbpu_whitehall}
	\end{subfigure}
	\hfill %выровнять
	\adjustbox{minipage=1.3em,valign=t}{\subcaption{}\label{fig:spbpu_whitehall-b}}%
	\begin{subfigure}[t]{\dimexpr.3\linewidth-1.3em\relax}
		\centering
		\includegraphics[width=.95\linewidth,valign=t]{my_folder/images//spbpu_whitehall_ligh}
	\end{subfigure}
	\hfill %выровнять
		\adjustbox{minipage=1.3em,valign=t}{\subcaption{}\label{fig:spbpu_whitehall-c}}%
	\begin{subfigure}[t]{\dimexpr.3\linewidth-1.3em\relax}
		\centering
		\includegraphics[width=.95\linewidth,valign=t]{my_folder/images//spbpu_whitehall_sculpture}
	\end{subfigure}%
\captionsetup{justification=centering} %центрировать
	\caption{Фотографии Белого зала СПбПУ \cite{spbpu-gallery}, в том числе: {\itshape a} --- со стороны зрителей; {\itshape b} --- со стороны сцены; {\itshape c} --- барельеф}\label{fig:spbpu_whitehall-three-photos}
\end{figure}

Далее можно ссылаться на три отдельных рисунка: \firef{fig:spbpu_whitehall-a}, \firef{fig:spbpu_whitehall-b} и \firef{fig:spbpu_whitehall-c}.
 % пример подключения 3х иллюстрации в одном рисунке
%
%На \firef{fig:spbpu_main_bld-two-photos} приведены две картинки под~общим номером и~названием.


\begin{figure}[!htbp]
	\adjustbox{minipage=1.3em,valign=t}{\subcaption{}\label{fig:spbpu_main_bld_entrance_autumn}}%
	\begin{subfigure}[t]{\dimexpr.5\linewidth-1.3em\relax} %разрешили выделить 0,5 стр в ширину на рисунок
		\includegraphics[height=0.20\textheight,valign=t]{my_folder/images//spbpu_main_bld_entrance_autumn} %высоту рисунка выставили как 0,3 от высоты наборного поля
	\end{subfigure}
%	\hfill %выровнять по ширине
	\adjustbox{minipage=1.3em,valign=t}{\subcaption{}\label{fig:spbpu_main_bld_whitehall}}%
	\begin{subfigure}[t]{\dimexpr.5\linewidth-1.3em\relax}%разрешили выделить 0,5 стр в ширину на рисунок
		\includegraphics[height=0.20\textheight,valign=t]{my_folder/images//spbpu_main_bld_whitehall}%высоту рисунка выставили как 0,3 от высоты наборного поля
	\end{subfigure}
\captionsetup{justification=centering} %центрировать
	\caption{Вид на главное здание СПбПУ \cite{spbpu-gallery}, включая: {\itshape a} --- вход со стороны парка осенью; {\itshape b}~--- окна Белого зала}\label{fig:spbpu_main_bld-two-photos}
\end{figure}

На \firef{fig:spbpu_main_bld_entrance_autumn} изображен вход со стороны парка СПбПУ осенью, а на \firef{fig:spbpu_main_bld_whitehall}~--- окна Белого зала.
 % пример подключения 2х иллюстраций в одном рисунке

Приведём пример табличного представления данных с записью продолжения на следующей странице на \taref{tab:long}.

%%% отладка longtable
%% 1) для контроля выхода таблицы за границы полей выставляем showframe в \geometry{}, см настройки
%% 2) используем \\* для запрета переноса определенной строки или средства из:
%% https://tex.stackexchange.com/q/344270/44348
%% 3) в крайнем случае для принудительного переноса таблицы на новую страницу используем \pagebreak после \\
\noindent % for correct centering
\begingroup
\centering
\small %выставляем шрифт в 12bp
\begin{longtable}[c]{|l|l|l|l|l|l|}
	\caption{Пример задания данных из \cite{Peskov2004} (с повтором для переноса таблицы на новую страницу)}%
	\label{tab:long}% label всегда желательно идти после caption
	\\
	\hline
	$G$&$m_1$&$m_2$&$m_3$&$m_4$&$K$\\ \hline
	1&2&3&4&5&6\\ \hline
	\endfirsthead%
	\captionsetup{format=tablenocaption,labelformat=continued} % до caption!
	\caption[]{}\\ % печать слов о продолжении таблицы
	\hline
	1&2&3&4&5&6\\ \hline
	\endhead
	\hline
	\endfoot
	\hline
	\endlastfoot
	$g_1$&0&1&1&0&1\\ \hline
	$g_2$&1&2&0&1&1\\ \hline
	$g_3$&0&1&0&1&1\\ \hline
	$g_4$&1&2&1&0&2\\ \hline
	$g_5$&1&1&0&1&2\\ \hline
	$g_6$&1&1&1&2&2\\ \hline
%
	$g_1$&0&1&1&0&1\\ \hline
	$g_2$&1&2&0&1&1\\ \hline
	$g_3$&0&1&0&1&1\\ \hline
	$g_4$&1&2&1&0&2\\ \hline \noalign{\penalty-5000} % способствуем переносу на следующую стр
	$g_5$&1&1&0&1&2\\ \hline
	$g_6$&1&1&1&2&2\\ \hline
%
	$g_1$&0&1&1&0&1\\ \hline
	$g_2$&1&2&0&1&1\\ \hline
	$g_3$&0&1&0&1&1\\ \hline
	$g_4$&1&2&1&0&2\\ \hline
	$g_5$&1&1&0&1&2\\ \hline
	$g_6$&1&1&1&2&2\\ \hline
%
	$g_1$&0&1&1&0&1\\ \hline
	$g_2$&1&2&0&1&1\\ \hline
	$g_3$&0&1&0&1&1\\ \hline
	$g_4$&1&2&1&0&2\\ \hline
	$g_5$&1&1&0&1&2\\ \hline
	$g_6$&1&1&1&2&2\\ \hline
%
	$g_1$&0&1&1&0&1\\ \hline
	$g_2$&1&2&0&1&1\\ \hline
	$g_3$&0&1&0&1&1\\ \hline
	$g_4$&1&2&1&0&2\\ \hline
	$g_5$&1&1&0&1&2\\ \hline
	$g_6$&1&1&1&2&2\\ \hline
%
	$g_1$&0&1&1&0&1\\ \hline
	$g_2$&1&2&0&1&1\\ \hline
	$g_3$&0&1&0&1&1\\ \hline
	$g_4$&1&2&1&0&2\\ \hline
	$g_5$&1&1&0&1&2\\ \hline
	$g_6$&1&1&1&2&2\\ \hline
%
	$g_1$&0&1&1&0&1\\ \hline
	$g_2$&1&2&0&1&1\\ \hline
	$g_3$&0&1&0&1&1\\ \hline
	$g_4$&1&2&1&0&2\\ \hline
	$g_5$&1&1&0&1&2\\ \hline
	$g_6$&1&1&1&2&2\\ \hline
\end{longtable}
\normalsize% возвращаем шрифт к нормальному
\endgroup
 % пример подключения таблицы на несколько страциц


\begin{table} [htbp]% Пример оформления таблицы
	\centering\small
	\caption{Пример представления данных для сквозного примера по ВКР \cite{Peskov2004}}%
	\label{tab:ToyCompare}		
		\begin{tabular}{|l|l|l|l|l|l|}
			\hline
			$G$&$m_1$&$m_2$&$m_3$&$m_4$&$K$\\
			\hline
			$g_1$&0&1&1&0&1\\ \hline
			$g_2$&1&2&0&1&1\\ \hline
			$g_3$&0&1&0&1&1\\ \hline
			$g_4$&1&2&1&0&2\\ \hline
			$g_5$&1&1&0&1&2\\ \hline
			$g_6$&1&1&1&2&2\\ \hline		
		\end{tabular}
%	\caption*{\raggedright\hspace*{2.5em} Составлено (или/и рассчитано) по \cite{Peskov2004}} %Если проведена авторская обработка или расчеты по какому-либо источнику	
	\normalsize% возвращаем шрифт к нормальному
\end{table}



%% please, before using, read the author guide carefully

\noindent % for correct centering
\begin{minipage}{\textwidth}
	\vspace{\mfloatsep} % интервал
	\centering\small
	\captionof{table}{Пример задания данных в табличном виде из \cite{Peskov2004} (с помощью окружения minipage)}%
	\label{tab:ToyCompare-Peskov-minipage}
	\begin{tabular}{|l|l|l|l|l|l|}
	\hline
	$G$&$m_1$&$m_2$&$m_3$&$m_4$&$K$\\
	\hline
	$g_1$&0&1&1&0&1\\ \hline
	$g_2$&1&2&0&1&1\\ \hline
	$g_3$&0&1&0&1&1\\ \hline
	$g_4$&1&2&1&0&2\\ \hline
	$g_5$&1&1&0&1&2\\ \hline
	$g_6$&1&1&1&2&2\\ \hline
	\hline
	\end{tabular}
\vspace{\mfloatsep} % интервал
\normalsize %восстанавливаем шрифт
\end{minipage}
 % пример подключения minipage

\noindent % for correct centering
\begin{minipage}{\textwidth}
	\centering
	\vspace{\mfloatsep} % интервал
	\includegraphics[keepaspectratio=true,scale=0.27] {my_folder/images/spbpu_new_bld_autumn}
	\captionof{figure}{Новый научно-исследовательский корпус СПбПУ \cite{spbpu-gallery} (с помощью окружения minipage)}\label{fig:spbpu-new-bld-autumn-minipage}
	\vspace{\mfloatsep} % интервал
\end{minipage}
 % пример подключения minipage




Вопросы форматирования текстово-графических объектов (окружений) не регламентированы в известных нам ГОСТах, поэтому предлагаем придерживаться следующих правил:

\begin{itemize}
	\item \textbf{полужирный текст} рекомендуем использовать только для названий стандартных окружений с нумерационной частью, например, для представления \textit{впервые}: \textbf{определение 1.1}, \textbf{теорема 2.2}, \textbf{пример 2.3}, \textbf{лемма 4.5};

	\item \textit{курсив} рекомендуем использовать только для выделения переменных в формулах, служебной информации об авторах главы (статьи), важных терминов, представляемых по тексту, а также для всего тела окружений, связанных с получением \textit{новых существенных результатов и их доказательством}: теорема, лемма, следствие, утверждение и другие.
\end{itemize}

 

По аналогии с нумерацией формул, рисунков и таблиц нумеруются и иные текстово-графические объекты, то есть включаем в нумерацию номер главы, например: теорема 3.1. для первой теоремы третьей главы монографии. Команды \LaTeX{} выставляют нумерацию и форматирование автоматически. Полный перечень команд для подготовки текстово-графических и иных объектов находится в подробных методических рекомендациях \cite{spbpu-bci-template-author-guide}. 


Для удобства авторов названия стандартных окружений, рекомендованных к использованию, приведены в \taref{tab:enum-std}, а в \taref{tab:enum-spbpu}  перечислены имена специально разработанных окружений для шаблонов SPbPU.

% и примеры их оформления на псевдокоде (см. \cite{cite-spbpu-bci}).


%https://tex.stackexchange.com/questions/2651/should-i-use-center-or-centering-for-figures-and-tables


	\begin{table} [htbp]% Пример записи таблицы с номером, но без отображаемого наименования
	\centering\small
	\caption{Стандартные окружения}%
	\label{tab:enum-std}
	 \begin{Spacing}{\Single} % Одинарный интервал между строками текста
	  \renewcommand*{\arraystretch}{1.5} % Полуторный интервал между ячейками таблицы
		\begin{tabular}{|l|p{11cm}|}
			\hline
			Название окружения&Назначение\\
			\hline
			\verb|center| &	центрирование, аналог команды \verb|\centering|, но с добавлением нежелательного пробела, поэтому лучше избегать применения \verb|center|\\ \hline
			\verb|itemize| &{перечисления, в которых нет необходимости нумеровать  пункты (немаркированные списки)} \\ \hline
			\verb|enumerate| & перечисления с нумерацией (немаркированные списки) \\ \hline
			\verb|refsection| & создание отдельных библиографических списков для глав \\ \hline
			\verb|tabular| & оформление таблиц \\ \hline
			\verb|table|   &{автоматическое перемещение по тексту таблиц, оформленных, например, с помощью \verb|tabular|, для минимизации пустых пространств} \\ \hline
			\verb|longtable| & оформление многостраничных таблиц \\ \hline
			\verb|tikzpicture| & создание иллюстраций с помощью пакета \verb|tikz| \cite{ctan-tikz} \\ \hline
			\verb|figure| &{автоматическое перемещение по тексту рисунков, оформленных например, с помощью \verb|tikz| или подключенных с помощью команды \verb|\includegraphics|, для минимизации пустых пространств} \\ \hline
			\verb|subfigure| & оформление вложенных рисунков в составе \verb|figure| \\ \hline
			\verb|algorithm| &{оформление псевдокода на основе пакета \verb|algorithm2e| \cite{ctan-algorithm2e}} \\ \hline
			\verb|minipage| & {оформление рисунков и таблиц без функций автоматического перемещения по тексту для  минимизации пустых пространств} \\ \hline
			\verb|equation| & {оформление выключенных (не встроенных в текст с помощью \verb|$...$|) одиночных формул на одной строке} \\ \hline
			\verb|multilined| &{оформление выключенных (не встроенных в текст с помощью \verb|$...$|) одиночных формул в несколько строк} \\ \hline
			\verb|aligned| &{оформление нескольких формул с выравниванием по символу \verb|&|.} \\ \hline
	\end{tabular}
	\end{Spacing}
%	\normalsize
	\end{table}

На базе пакета \verb|tikz| разработано большое количество расширений \cite{ctan-tikz}, например, \verb|tikzcd|, которые мы рекомендуем использовать для оформления иллюстраций.

	\begin{table} [htbp]% Пример записи таблицы с номером, но без отображаемого наименования
	\centering\small
	\caption{Специальные окружения}%
	\label{tab:enum-spbpu}
		\begin{tabular}{|l|l|}
			\hline
			Название окружения & Текстово-графический объект\\
			\hline
			\verb|abstr|	 & реферат (abstract) \\ \hline
			\verb|m-theorem| & теорема \\ \hline
			\verb|m-corollary| & следствие \\ \hline
			\verb|m-proposition| & утверждение \\ \hline
			\verb|m-lemma|   & лемма \\ \hline
			\verb|m-axiom| & аксиома \\ \hline
			\verb|m-example| & пример \\ \hline
			\verb|m-definition| &  определение \\ \hline
			\verb|m-condition| & условие \\ \hline
			\verb|m-problem| & проблема \\ \hline
			\verb|m-exercise| & упраженение \\ \hline
			\verb|m-question| & вопрос \\ \hline
			\verb|m-hypothesis| & гипотеза \\ \hline
		\end{tabular}
	\normalsize
\end{table}

В случае, если авторам потребовалось новое окружение, то создать его можно в файле в файле \texttt{my\_fol\-der/{}my\_set\-tings.tex} согласно правилам, приведённым ниже.

\begin{enumerate}[1.]
	\item Для перехода в режим создания окружений следует указать:
	\begin{itemize}
		\item \verb|\theoremstyle{myplain}| --- окружения с доказательствами или аксиомами
		\item \verb|\theoremstyle{mydefinition}| --- окружения, не связанные с доказательствами или аксиомами.
	\end{itemize}
	\item В команде создания окружения следует ввести краткий псевдоним (\verb|m-new-env|) и отображаемое в pdf имя окружения (\verb|Название_окружения|):
	\begin{itemize}
		\item \texttt{\textbackslash{}newtheorem\{m-new-env-second\}\{Название\_окруже\-ния\}\-[chap\-ter]}.
	\end{itemize}
\end{enumerate}


%\begin{m-new-env-first}
%	Тест первого пользовательского окружения
%\end{m-new-env-first}
%
%\begin{m-new-env-second}
%	Тест второго пользовательского окружения
%\end{m-new-env-second}
 % список некоторых окружений


\begin{m-theorem}[о чем-то конкретном] %при необходимости в [] можно записать название теоремы или убрать его
	\label{th:ex}
	% \index только для принятых работ
	% шаблон записи теоремы в Предметный указатель
	\index[ru]{теорема!название\_теоремы или о чём} %ключевое слово <<теорема>> не менять
	\index[en]{theorem!1-3 words for detail or description}
	% пример записи алгоритма в Предметный указатель
	\index[ru]{теорема!о неполноте}
	\index[en]{theorem!about incompleteness}
	% пример записи алгоритма в Предметный указатель
	\index[ru]{теорема!о жизни}
	\index[en]{theorem!about life}
	Текст теоремы полностью выделен курсивом. Допустимо математические символы не выделять курсивом, если это искажает их значения. Используется абзацный отсуп, так как ``Абзацы в тексте начинают отступом'' в соответствии с ГОСТ 2.105--95. Название теоремы допустимо убрать. Доказательство окончено.
\end{m-theorem}
Доказательство теоремы \ref{th:ex}, леммы, утверждений, следствий и других подобных окружений (в последнем абзаце) завершаем предложением в котором сказано, что доказательство окончено. Например, доказательство теоремы \ref{th:ex} окончено.

Тело доказательства не выделяется курсивом.
Тело следующих окружений также не выделяется сплошным курсивом: определение, условие, проблема, пример, упражнение, вопрос, гипотеза и другие.
 %пример оформления теоремы


\begin{m-definition}[термин] %при необходимости в [] можно записать название определения или убрать его
	\label{def:ex}
	% \index только для принятых работ
	% шаблон записи определения в Предметный указатель
	\index[ru]{название\_определения!1-3 уточняющих слова или~ничего}
	\index[en]{definition\_title!1-3 words for detail or~without "!-part}
	% пример записи определения в Предметный указатель
	\index[ru]{и-тест!хороший!наилучший}
	\index[en]{i-test!good!best}
	% пример записи определения в Предметный указатель
	\index[ru]{и-тест!замкнутый}
	\index[en]{i-test!closed}
	В тексте определения только {\itshape важные термины} выделяются курсивом. Если определение носит лишь вспомогательный характер, то допустимо не использовать окружение \texttt{m-definition}, представляя текст определения в обычном абзаце. Ключевые термины при этом обязательно выделяются курсивом.
\end{m-definition}
 %пример оформления определения


Вместо теоремо-подобных окружений для вставки небольших текстово-графических объектов иногда используются команды. Типичным примером такого подхода является команда \verb|\footnote{text}|\footnote{Внимание! Команда вставляется непосредственно после слова, куда вставляется сноска (без пробела). Лишние пробелы также не указываются внутри команды перед и после фигурных скобок.}, где в аргументе \verb|text| указывают текст \textit{подстрочной ссылки (сноски)}.В них \textit{нельзя добавлять веб-ссылки или цитировать литературу}. Для этих целей используется список литературы. Нумерация сносок сквозная по ВКР без точки на конце выставляется в шаблоне автоматически, однако в каждом приложении к ВКР нумерация, зависящая от номера приложения, выставляется префикс <<П>>, например <<П1.1>> --- первая сноска первого приложения. 




%\FloatBarrier % заставить рисунки и другие подвижные (float) элементы остановиться


\section{Выводы} \label{ch2:conclusion}

Текст заключения ко второй главе. Пример ссылок \cite{Article,Book,Booklet,Conference,Inbook,Incollection,Manual,Mastersthesis,Misc,Phdthesis,Proceedings,Techreport,Unpublished,badiou:briefings}, а также ссылок с указанием страниц, на котором отображены те или иные текстово-графические объекты  \cite[с.~96]{Naidenova2017} или в виде мультицитаты на несколько источников \cites[с.~96]{Naidenova2017}[с.~46]{Ganter1999}. Часть библиографических записей носит иллюстративный характер и не имеет отношения к реальной литературе. 

Короткое имя каждого библиографического источника содержится в специальном файле \verb|my_biblio.bib|, расположенном в папке \verb|my_folder|. Там же находятся исходные данные, которые с помощью программы \texttt{Biber} и стилевого файла \texttt{Biblatex-GOST} \cite{ctan-biblatex-gost} приведены в списке использованных источников согласно ГОСТ 7.0.5-2008.
Многообразные реальные примеры исходных библиографических данных можно посмотреть по ссылке \cite{ctan-biblatex-gost-examples}.

Как правило, ВКР должна состоять из четырех глав. Оставшиеся главы можно создать по образцу первых двух и подключить с помощью команды \verb|\input| к исходному коду ВКР. Далее в приложении \ref{appendix-MikTeX-TexStudio} приведены краткие инструкции запуска исходного кода ВКР \cite{latex-miktex,latex-texstudio}.

В приложении \ref{appendix-extra-examples} приведено подключение некоторых текстово-графических объектов. Они оформляются по приведенным ранее правилам. В качестве номера структурного элемента вместо номера главы используется <<П>> с номером главы. Текстово-графические объекты из приложений не учитываются в реферате.



%% Вспомогательные команды - Additional commands
%
%\newpage % принудительное начало с новой страницы, использовать только в конце раздела
%\clearpage % осуществляется пакетом <<placeins>> в пределах секций
%\newpage\leavevmode\thispagestyle{empty}\newpage % 100 % начало новой страницы